Let $X$ be a smooth separated scheme$/\Bbbk$. A. Beilinsun and S.Lichtenbaum conjectured around 1982--1987 that there should exist certain complexes $\ZZ(n)$, $n \in \NN$ of
schemes in Zariski topology on $\Sm /\Bbbk$ which have the following properties:
\begin{enumerate}
	\item
			$\ZZ(0)$ is the constant sheaf $\zZz$.
	\item
			$\ZZ(1)$ is the complex $0^{\str}[-1]$:
			\[
					\cdots\to 0\to \underset{1}{\OoO^{\str}}\to0\to\cdots.
			\]
	\item
			For every field $F / \Bbbk$,
			\[
					\HH_{\Zar}^n(\fFf, \ZZ(n)) \coloneq H^n(\ZZ(n)(\Spec \fFf)) = K_{n}^{\Mil}(\fFf)
			\]
			where $K_n^{\Mil}(\fFf)$ is the $n$-th Milnor $K$-group of $K_0^{\Mil}(\fFf)=2$,
			$K_1^{\Mil}(\fFf)=\fFf^{\times}$, $K_2^{\Mil}(F)=K_2(\fFf)$.
	\item
			$\HH_{\Zar}^{2 n}(X,
			\ZZ(n))=CH^n(x)$, where
			\[
					CH^n(x)=\ZZ\set{\text {cycles of }\codim=n} / \text { rat. equi.}
			\]
			is the $n$-th Chow group.
	\item
			There is a natural spectral sequere with
			\[
					E_2^{p, q} = \HH_{\Zar}^p(X,\ZZ(q))\Rightarrow K_{2q-p}(X),
			\]
			where $K_n(X)=\pi_{n+1}(B Q(\VECT(X)), 0)$ is the Quillen $K$-theory. Tensoring with
			$\QQ$, the spectral sequence degenerates and one has $$ \HH_{\Zar}^i(X,
			\ZZ(n))_{\QQ}=\gr_r^n K_ {2 n-i}(X)_{\QQ}, $$ where $\gr_r^n$ are quotient of the
			$\gamma$-filtration.
\end{enumerate}

\subsection{motivic cohomology}

We denote by $H^{p,q}=\HH_{\Zar}^p(X, \ZZ(q))$ the motivic cohomology of $X$. It
satisfies the cancellation property: $$ H^{p, q}(X \times \GG_{m}, \ZZ)=H^{p, q}(X, \ZZ)
\oplus H^{p-1, q-1}(X, \ZZ). $$

It turns out that the group remains unchanged if we replace Zariski topology by Nisnevich
topology. If one uses étale topology, the corresponding cohomologies are denoted by
$H_L^{p, q}(X, \ZZ)$, the Lichtenbanm motivic cohomologies. It admits the following
comparison
\[
		H_l^{p, q}(X, \ZZ/n)=H_{\et}^p(X, \underset{\uparrow=\mu_n^{\otimes q}}{\ZZ/n(q)}).
\]
f $\chi\Bbbk\nmid n$, The Beilinson--Lichtenbaum conjecture, proved by V.Voevodsky in
2011 , states that $$ H^{p, q}(X, \ZZ / n) \rightarrow H_L^{p, q}(X, \ZZ / n) $$ is an
isomorphism if $p \leq q$ and monomorplisin for $p=q+1$. So we obtain $$ H^{p, q}(X, \ZZ
/ n)=H_{\et}^p(X, \ZZ / n(q)) $$ if $p \leq q$. If we take $X=\Spec(\Bbbk)$, this is the
Milnor Conjecture, proved by V. Voerosky in 1996:
\[
		H^{p, p}(\Bbbk, \ZZ / n)=K_p^{\Mil}(\Bbbk) / n=H_{\et}^p(\Bbbk, \ZZ / n(p)).
\]
$$
H^{p, q}(\Bbbk, \ZZ / n)=
\begin{cases}
	0                                  & p>q \\
	H^{p, p}(\Bbbk, \ZZ / n)\tau^{q-p} & p<q
\end{cases}
$$
where $\tau \in M_n(\Bbbk)=H^{0, 1}(\Bbbk, \ZZ / n)$ is the primitive root.

Unlike finite coefficients, the $H^{p,q}(\Bbbk, \ZZ)$ is quite hard to compute for small
$p$. The open question is the Beilinson--Somle vanishing conjecture: $$ H^{p, q}(\Bbbk,
\ZZ)=0\quad \text { if } p<0. $$

If $\chi\Bbbk=0$, this is known for number fields, function fields of genus $0$ curves
over number field and their inductive limits. If $\chi\Bbbk>0$, this is known for finite
fields and global fields.

The motivic cohomologies could be realized in a triangulated category $\DM(\Bbbk)$, such
that $$ H^{p, q}(X, \ZZ)=H_{\DM}(\ZZ(X), \ZZ(q)[p]) $$ where $\ZZ(X)$ is the motive of
$X$ and $\ZZ(q)[p]=\GG_m^{n q}[p-q]$. Moreover, we can define the motivic homology $H_{p,
		q}(X, \ZZ)$ as $$ H_{p, q}(X, \ZZ)=H_{\DM}(z(q)[p], \ZZ(X)). $$

From the aspect of motives, we can derive theorem of all (co)homologies which can be
represented in $\DM$. The main derives are the following:
\begin{enumerate}
	\item
			If $E \rightarrow X$ is an $\AA^n$-bundle, we have $\ZZ(E)=\ZZ(X)$ in $\DM$.
	\item
			If $\set{U, V}$ is an open covering of $X$, we have the Mayer-Vietoris sequence in $\DM$. $$ \cdots
			\rightarrow \ZZ(U \cap V) \rightarrow \ZZ(U) \oplus \ZZ(V) \rightarrow \ZZ(X) \rightarrow \cdots.
			$$
	\item
			If $Y \subseteq X$ is a closed embedding of $\codim$ $c$ in $\Sm_{\Bbbk}$, we have a Gysin triangle
			$$ \ZZ(X \setminus Y) \rightarrow \ZZ(X) \rightarrow \underset{=\ZZ(Y)(C)[2c]}{\ZZ(Y) \otimes
				\ZZ(c)[2c]} \rightarrow \cdots. $$
	\item
			For any vector bundle $\eEe$ of rank $n$ on $X$, we have the projective bundle formula: $$
			\ZZ(\mathbb{P}(\eEe))=\bigoplus_{i=0}^n \ZZ(X)(i)[2 i], $$ which defines the chern classes of
			$\eEe$.
	\item
			For proper smooth $x$, the $\ZZ(x)$ has a strung dual $\ZZ(x)(-\dim x)[-2 \dim x]$ in $D M$, which
			implies the Poincare duality $$ H^{p, q}(X, \ZZ)=H_{2 \dim X-p, 2 \dim X-q}(X, \ZZ). $$
\end{enumerate}

\subsection{Outline Of The Lesson}

The course consists of the following:
\begin{enumerate}
	\item
			Intersection theory
	\item
			Milnor $K$-thpory and cycle modules
	\item
			The (effective and stable) (allegories of motives and basic tools.
	\item
			Computation of $H^{n, n}$ and $H^{2 n, n}$
\end{enumerate}